\chapter{Related Work} % (fold)
\label{cha:background}
Maintaining balance is one of the key challenges inherent to the bipedal form. Most traditional approaches have been based on the concept of constantly maintaining balance \cite{Kajita:2001fk,Sugihara:2002kq}.  

\section{Zero-Moment Point} % (fold)
\label{sec:existing_strategies}
The most popular techniques to achieve walking have been trajectory generation and control strategies based on the Zero-Moment Point (ZMP) criterion \cite{Vukobratovic:2004wy}. The ZMP defines a point on the ground where the forces acting on a biped do not produce a moment about the axes parallel to the ground plane. If the ZMP is kept within the region of foot support, the biped is stable. This stability criterion can be used to compute ZMP-stable trajectories offline that maximize the stability margin by maximizing the distance from the ZMP to the boundary of the support region. In the on-line phase, the stable trajectories are tracked online to execute walking \cite{HuangEtAlTRA2001}.  ZMP-based trajectories can also be generated on-line \cite{KajitaEtAlICRA2003, TakenakaEtAlIROS2009}. One of the biggest drawbacks of this approach is the resulting trajectory does not provide any strategy to respond to disturbances due to uneven terrain or unexpected forces. Typically, these strategies \cite{Kajita:1997vr,Sugihara:2002kq} are also energetically inefficient since they are constantly trying to maintain balance by keeping the ZMP point within the region of foot support. Furthermore, the resulting gait does not utilize the natural dynamics of the system and consequently does not look human-like.

An alternative approach, first proposed by McGeer \cite{McGeer:1990uk}, introduced a unique class of legged robots known as passive dynamic walkers \cite{Collins:2005vp}. This approach takes advantage of the natural dynamics of the biped structure, and is capable of maintaining a stable gait cycle without active control effort. Fully passive mechanisms walk on an inclined surface so that the mechanism is powered by gravity alone \cite{Spong:1999vk}. In addition to producing highly efficient walk, the gait patterns generated using this approach appear more human-like in comparison to ZMP-based control. However, passive dynamic walkers lack robustness to perturbations due to very narrow regions of attraction.
% section existing_strategies (end)

\section{Foot Placement Strategies} % (fold)
\label{sec:related_work}
Recently, an alternative problem formulation focusing on restoring balance has been proposed. The Foot Placement Estimator (FPE), introduced by Dwight et al \cite{Wight:2008ii} formulates an approach to restore balance by controlling swing foot position during the gait cycle. By using the conservation of angular momentum, the FPE equation determines the location on the ground where the total energy of an unstable biped after swing foot impact is equal to the peak potential energy. If a step is taken before the FPE location, the post impact energy of the system causes the biped to fall over. Conversely, stepping beyond the FPE location on the ground causes the biped to fall back onto the hind leg.

The solution to the FPE equation itself can be used as a recovery mechanism (i.e. in the face of a destabilizing disturbance) with existing ZMP-based strategies. Alternatively, it can be used to increase the narrow regions of attraction which plague minimally actuated passive dynamic walkers \cite{Goswami:1996gn,Asano:2000wi,Kuo:1999tn}. The key concept here is that the FPE-based integration would require minimal joint actuation only to align a swing food appropriately to recover from a potential fall. As shown in \cite{Wight:2008ii,Wight:2008vt}, FPE can also be extended to form complete gait cycles to achieve dynamically stable walking. However, there are several key assumptions which are violated when attempting to implement this approach on a physical 3D robot. Namely, the theory presented in the derivations assume that the legs are massless and it only deals with the 2D dynamics in the sagittal plane.

The capture point concept, developed by Pratt et al. \cite{Pratt:2006vy}, is conceptually similar to the FPE. While the derivation of FPE is based on a simple compass biped model with fixed parameters, the capture point theory was derived using complex motion models which included using a flywheel body to control/offset any disturbances through the use of rotational inertia. Ultimately, the simplicity of the model allowed the FPE theory to be extended to complete gait cycles, while the work presented by Pratt et al simply solved the problem of lateral stabilization \cite{Wight:2008ii}.

Recently, a more comprehensive approach using the capture point for foot placement as a means to develop full walking control strategies has been proposed. De Boer \cite{DeBoer:2012wp} focused on the ground/foot interaction to develop a robust and energetically efficient walking control strategy for a force-controlled compliant lower-body biped.  While this approach is philosophically similar to the idea behind FPE, there are several key differences. Our approach uses simple local controllers to form complete gait cycles, and can be used on position-controlled joints without any complex actuation systems. The capturability framework demonstrated in \cite{Pratt01092012} used separate controllers for the swing and stance legs whereas our approach uses a single global differential kinematic resolution for whole body motion control. This simplifies our approach as multiple controllers do not need to be tuned to achieve walking.
% section related_work (end)

% chapter background (end)