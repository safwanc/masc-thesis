\chapter{Introduction} % (fold)
\label{cha:introduction}
Bipedal locomotion is a highly active research area within the field of humanoid robotics. As robotics continue to push towards becoming more mainstream, the ability for a robotic system to navigate an environment designed for humans is essential for them to be useful. However, bipedal locomotion is a very challenging control problem due to the natural instability of a dynamic and nonlinear system. Over the past 40 years, most control strategies have focused around using a measure of balance to come up with walking control algorithms. However, the downside thus far has been energetically inefficient gait which is very ``robot-like''. 

In this thesis, we present a novel walking control strategy which is energetically efficient and robust to external disturbances. However, a key challenge in validating most control approaches in the field of humanoid robotics is validating simulations with physical hardware. To this end, this thesis outlines the construction of a 14 degree-of-freedom (DOF) lower body humanoid robot for the purposes of research in bipedal locomotion. The ultimate goal is to apply the walking control strategy on the bipedal robot and validate the effectiveness and usefulness of the presented walking control strategy. 

\section{Contributions} % (fold)
\label{sec:contributions}
At the very core of the proposed walking control strategy is the Foot Placement Estimator (FPE) algorithm which has been presented for 2D bipedal robots. Simply put, the FPE algorithm determines where an unstable bipedal robot must step to regain balance. The elegance of this approach lies in the fact that it can be extended to form complete gait cycles in the 2D case using simple PD controllers and a state machine. 

In this thesis, the 2D FPE theory is extended to form complete gait cycles for a 3D bipedal robot in simulation and ultimately, on physical hardware. Extending the 2D theory presents its own unique set of challenges which are addressed in part using prioritized whole body motion control. Ultimately, the efficacy of the proposed walking control strategy is demonstrated in simulation. The physical 14 DOF bipedal robot is constructed but due to time limitations, only the validity of the motion control framework is demonstrated on physical hardware. 
% section contributions (end)

\section{Outline} % (fold)
\label{sec:outline}
The remainder of this thesis is organized as follows. Chapter \ref{cha:background} outlines existing walking control strategies prevalent in literature today. While most control strategies use a dynamic measure of balance to form complete gait cycles, some newer approaches are similar to the strategy presented in this thesis. That is, these control strategies aim to use foot placement as a means to form robust and energetically efficient gait for bipedal locomotion. 

One of the key challenges in developing physical hardware for bipedal locomotion research is the impact of small changes in the electromechanical design on the overall stability of the system. In order to effectively develop a system with adequate drivetrain performance to achieve walking, a toolchain is developed to streamline the process of designing a mechanical prototype in CAD and immediately analyzing its behaviour in dynamic simulations. The electromechanical design workflow using the toolchain is presented in Chapter \ref{cha:toolchain}. 

Extending the current 2D FPE theory to the 3D case is presented in Chapter \ref{cha:simulations}. A prioritized whole body motion control scheme is presented in order to simultaneously use the FPE algorithm to achieve walking and maintain stability throughout the entire gait cycle. The 2D FPE theory and the proposed extension to 3D are both verified in dynamic simulations. 

Next, the electromechanical design of a 14 DOF lower body humanoid robot is presented in Chapter \ref{cha:design}. Human gait is used as an approximate starting point for actuators and drivetrain component selection from local vendors. The forward dynamics and 3D visualization for the final prototype design is generated using the toolchain outlined in in Chapter \ref{cha:toolchain}. This is used to test control strategies on the 14 DOF lower body humanoid platform in simulation prior to implementing it on the physical hardware. 

Experimental validation of the drivetrain dynamics and whole body motion control frameworks are validated on the physical hardware in Chapter \ref{cha:experiments}. The actuator dynamics models in simulation are verified to match the physical DC motors through hardware-in-the-loop (HIL) simulations. The whole body motion control framework used for extending the FPE theory to 3D is also verified on physical hardware using HIL.  

Lastly, conclusions and future work regarding the proposed walking control strategy and developed physical hardware are presented in Chapter \ref{cha:conclusion}. 
% section outline (end)

% chapter introduction (end)