\chapter{Introduction} % (fold)
\label{cha:introduction}
Bipedal locomotion is a highly active research area within the field of humanoid robotics. It enables robots to navigate a wide range of environments required for most practical real-world applications. However, bipedal locomotion is a very challenging control problem due to the postural instability of the bipedal form.

A robust and energetically efficient walking control strategy is presented in this thesis. One of the many challenges with novel control strategies is validating simulations with physical hardware. To this end, this thesis outlines the construction of a 14 degree-of-freedom (DOF) lower body humanoid robot for the purposes of research in bipedal locomotion. The ultimate goal is to validate the effectiveness and usefulness of the presented walking control strategy by testing it on the physical biped hardware. 

At the core of the proposed walking control strategy is the Foot Placement Estimator (FPE) algorithm which has been presented for 2D bipedal robots \cite{Wight:2008vt}. The FPE algorithm determines where an unstable biped must step to restore balance. One of the key novelties of the FPE theory is the wide range of applicability of the results. It can be used to augment existing walking control strategies or be extended to form dynamically stable gait cycles \cite{Wight:2008ii}.

The most popular technique to achieve walking thus far has been through control strategies based on the Zero-Moment Point (ZMP) criterion \cite{Vukobratovic:2004wy}. ZMP-stable trajectories are often computed offline so that the robot can can track/reconfigure itself online to execute walking. Typically, these strategies \cite{Kajita:1997vr,Sugihara:2002kq} are energetically inefficient since they are constantly trying to maintain balance by keeping the ZMP point within the region of foot support. Furthermore, the resulting gait does not utilize the natural dynamics of the system and consequently does not look ``human-like''. 

More recent research in the field of bipedal locomotion has gained significant traction due to its ability of maintaining a stable gait cycle without active control efforts. This pioneering research by McGeer \cite{McGeer:1990uk} introduced a unique class of legged robots known as passive dynamic walkers \cite{Collins:2005vp}. This new class of legged robots are designed to walk on a on an inclined surface so that the passive mechanism is powered by gravity alone \cite{Spong:1999vk}. In addition to producing highly efficient walk, the gait patterns generated using this approach ar much more human-like in comparison to ZMP-based control. However, passive dynamic walkers lack robustness to perturbations due to very narrow regions of attractions. 

These two opposing strategies illustrate the usefulness and versatility of the FPE based approach presented in \cite{Wight:2008vt}. The solution to the FPE equation itself can be used as a recovery mechanism (i.e. in the face of a destabilizing disturbance) with existing ZMP-based strategies. Alternatively, it can be used to increase the narrow regions of attraction which plague minimally actuated passive dynamic walkers \cite{Goswami:1996gn,Asano:2000wi,Kuo:1999tn}. The key concept here is that the FPE-based integration would require minimal joint actuation only to align a swing food appropriately to recover from a potential fall. 

\section{Contributions} % (fold)
\label{sec:contributions}

The 2D FPE algorithm is extended to form complete gait cycles for a 3D bipedal robot in simulation and ultimately, on physical hardware. Extending the 2D theory presents its own unique set of challenges which are addressed in part by using a motion control framework and selecting a 2D plane to solve the FPE algorithm. Ultimately, the efficacy of the proposed walking control strategy is demonstrated through side stepping and forward walking in dynamic simulations. 

The development of a 14 DOF bipedal robot for the purposes of research in bipedal locomotion is also presented. A rapid prototyping toolchain is developed to streamline the iterative electromechanical design process. Drivetrain component selection is performed using gait estimation and accounting for actuator dynamics. The physical 14 DOF bipedal robot is constructed and the validity of the motion control framework and actuator dynamics models are demonstrated on the physical hardware. 
% section contributions (end)

\section{Outline} % (fold)
\label{sec:outline}
The remainder of this thesis is organized as follows. Chapter \ref{cha:background} outlines existing walking control strategies in the robotics literature. While most control strategies use a dynamic measure of balance to form complete gait cycles, some newer approaches are similar to the strategy presented in this thesis. These control strategies aim to use foot placement as a means to form robust and energetically efficient gait for bipedal locomotion. 

The initial electromechanical design and development of a 14 DOF lower body humanoid robot (biped) is presented in Chapter \ref{cha:design}. Human gait is used as an approximate starting point for actuators and drivetrain component selection. The forward dynamics and 3D visualization for the final prototype design is generated using the toolchain presented in Chapter \ref{cha:toolchain}. This is used to test control strategies for the biped in simulation prior to implementing it on the physical hardware.

One of the key challenges in developing physical hardware for bipedal locomotion research is the impact of small design changes on the overall stability of the system. A toolchain is developed in Chapter \ref{cha:toolchain} to help improve the initial design and ultimately produce a system with adequate drivetrain performance. The toolchain streamlines the process of designing a mechanical prototype in Computer-Aided-Design (CAD) software and immediately analyzing its behaviour in dynamic simulations.

The extension of the 2D FPE theory to the 3D case is presented in Chapter \ref{cha:simulations}. A motion control framework is presented in order to simultaneously use the FPE algorithm to achieve walking and maintain stability throughout the gait cycle. The 2D FPE theory and the proposed extension to 3D are verified through dynamic simulations. 

Experimental validation of the actuator dynamics and motion control framework are validated on the physical hardware in Chapter \ref{cha:experiments}. The actuator dynamics models are verified to match the physical DC motors through hardware-in-the-loop (HIL) simulations. The motion control framework used for extending the FPE theory to 3D is also verified on physical hardware using HIL.  

Lastly, conclusions and future work regarding the proposed walking control strategy and developed physical hardware are presented in Chapter \ref{cha:conclusion}. 
% section outline (end)

% chapter introduction (end)